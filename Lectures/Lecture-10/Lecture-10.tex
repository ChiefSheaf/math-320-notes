\clearpage

\begin{nquote}{}
	``How about the super on steroids version of the add and subtract trick?" - Dr. Philip Loewen, 09/27/2023
\end{nquote}

\section{Completeness}
``The property that makes \(\R\) better than \(\Q\)."

\subsection{Cauchy sequences}
\begin{ndef}{: Cauchy sequences}
	Statement 1: A sequence \((x_n)\) is called \emph{\textbf{Cauchy}} when for all \(\eps>0\), there exists \(N\in\N\) such that for all \(m,n\geq N\), we have \(|x_m-x_n|<\eps\).
	
	\medskip
	
	Statement 2: An equivalent way of saying this is that for all \(\eps>0\), there exists \(N\in\N\) such that for all \(n\geq N\) and \(p\in\N\), we have \(|x_{n+p}-x_n|<\eps\).
\end{ndef}
\begin{proposition}
	Every convergent sequence is Cauchy.
\end{proposition}
\begin{proof}
	We begin by picking a convergent sequence: let \((x_n)\) converge to \(\hat{x}\). Estimate 
	\begin{align*}
		|x_n-x_m|=&|(x_n-\hat{x})+(\hat{x}-x_m)|\\
				 \leq&|x_n-\hat{x}|+|x_m-\hat{x}|.
	\end{align*}
	To show that this sequence Cauchy (Statement 1), let \(\eps>0\) be given and use definition of \(x_n\to\hat{x}\) with \(\eps'=\displaystyle\frac{\eps}{2}\) to get \(N\in\N\) such that \(|x_k-\hat{x}|<\eps'\) whenever \(k>N\). This \(N\) works in statement 1, since from what we have shown above, \(m,n\geq\N\implies |x_m-x_n|<\eps'+\eps'=\eps\). 
\end{proof}
\begin{corollary}
	Any sequence that is \emph{not} Cauchy \emph{must} diverge.
\end{corollary}
\begin{proof}
	Contrapositive of the statement above; in general this is a great approach when proving divergence.
\end{proof}
\begin{ntheorem}{: Metric completeness}
	Every Cauchy sequence converges (to a real limit) in \(\R\).
\end{ntheorem}
The proof for this is something we will revisit after we have a bit more machinery, which we will now develop.
\subsection{Bounded sets}
\begin{ntheorem}{: Order completeness}
	Given any non-empty \(\mc{S}\subseteq\R\), let \(\mc{A}:=\{a\in\R\st\text{for all}~x\in \mc{S},~a\leq x\}\), \(\mc{B}:=\{b\in\R\st\text{for all}~x\in \mc{S},~x\leq b\}\), then:
	\begin{enumerate}[(a)]
		\item Either \(\mc{A}=\emptyset\) or \(\mc{A}=(-\infty,\alpha]\) for some \(\alpha\in\R\).
		
		\item Either \(\mc{B}=\emptyset\) or \(\mc{B}=[\beta,\infty)\) for some \(\beta\in\R\).
	\end{enumerate}
\end{ntheorem} 
We say that \(\mc{S}\) is \emph{bounded above} when \(\mc{B}\neq\emptyset\), and call each \(b\in \mc{B}\) an \emph{upper bound for \(\mc{S}\)}.

\medskip

Similarly, \(\mc{S}\) is \emph{bounded below} when \(\mc{A}\neq\emptyset\) each \(a\in \mc{A}\) is a \emph{lower bound for \(\mc{S}\)}. Just the word ``bounded" means ``bounded above" \emph{and} ``bounded below."

\medskip

We now define one of the most important concepts of this course:
\begin{ndef}{: Supremum}
	When \(\mc{B}\neq\emptyset\), we call \(\beta\) the \emph{\textbf{supremum}} of \(\mc{S}\), i.e., \(\beta=\text{sup}(\mc{S})\).
	
	\medskip
	
	Useful characterization:
	\begin{enumerate}[(i)]
		\item For all \(x\in \mc{S}\), \(x\leq \beta\) is the same as saying ``\(\beta\) is an \emph{upper bound} for \(\mc{S}\)."
		
		\item For all \(\gamma<\beta\), there exists \(x\in \mc{S}\) such that \(\ga<x\), which is the same as saying ``nothing \emph{less than} \(\beta\) is an upper bound." This is why another name for the supremum is \emph{the least upper bound}.
	\end{enumerate}
\end{ndef}
Similarly, we define:
\begin{ndef}{: Infimum}
	When \(\mc{A}\neq\emptyset\), \(\alpha=\text{inf}(\mc{S})\) is the \emph{\textbf{infimum}} or \emph{the greatest lower bound} of \(S\).
\end{ndef}
\subsection{Monotonic sequences}
\begin{ntheorem}{: Monotonic sequence property}
	Given any sequence \((x_n)\) with \(x_1\leq x_2\leq x_3\leq\dots\), either \(x_n\to\infty\) \emph{or} \(x_n\) converges to a real limit.
\end{ntheorem}
\begin{note}
	When we say that \(x_n\to\infty\), we are saying more than just ``the sequence diverges", we are commenting on specifically \emph{how} it diverges.
\end{note}
\begin{note}
	We will prove all these theorems at some point in this course, however, right now we will take them for granted.
\end{note}
\begin{note}[Linkages]
	These 3 viewpoints on completeness contain equivalent information; each one implies the others.
\end{note}
Going back to metric completeness, we show one of these linkages:
\begin{ntheorem}{}
	Metric completeness (which says Cauchy sequences must converge) implies order completeness (If \(\mc{S}\neq\emptyset\) is bounded above, sup\((\mc{S})\) exists.)
\end{ntheorem}
\begin{proof}
	Let \(\mc{S}\subseteq\R\) be non-empty; define \(\mc{B}=\{b\in\R\st\text{for all}~s\in \mc{S},~s\leq b\}\). Assume \(\mc{B}\neq\emptyset\) and define sequence \(b_n=\operatorname{min}\left\{\mc{B}\cap\left\{\displaystyle\frac{k}{2^n}:~k\in\Z\right\}\right\}\).  This is a Cauchy sequence (we just assert this but this is certainly something we would have to prove in a proof usually.) Thus, \(\beta=\displaystyle\lim_{n\to\infty}b_n\) will have the properties defining sup\((\mc{S})\). For each fixed \(n\), we have 
	\begin{enumerate}[(i)]
		\item \(b_n-\displaystyle\frac{1}{2^n}\notin \mc{B}\implies\) there exists \(s_n\in \mc{S}\) such that \(s_n>b_n-\displaystyle\frac{1}{2^n}\).
		
		\item \(b_{n+1}\leq b_n\) (minimum over a larger set of points.)  
		
		\item Using \(b_{n+1}\in \mc{B}\), we have \(b_{n+1}\geq s_n\) for \(s_n\) above. Using (ii),
		\begin{equation*}
			b_{n+1}\geq s_n>b_n-\displaystyle\frac{1}{2^n}\iff 0\leq b_n-b_{n+1}<\displaystyle\frac{1}{2^n}.
		\end{equation*}
	\end{enumerate}
	Now, we estimate
	\begin{align*}
		|b_{n+p}-b_n|=&b_n-b_{n+p}\\
					 =&(b_n-b_{n+1})+(b_{n+1}-b_{n+2})+\dots+(b_{n+p-1}-b_{n+p})\\
					 <&\frac{1}{2^n}+\frac{1}{2^{n+1}}+\dots+\frac{1}{2^{n+p-1}}+\frac{2}{2^n}.
	\end{align*}
	This is the key to showing \((b_n)\) is Cauchy.
\end{proof}