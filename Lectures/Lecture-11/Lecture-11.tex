\clearpage
\begin{nquote}{: Dr. Zahl special}
	``I won't intentionally say something wrong, but I might; I hit a car on my way to work today so I'm a bit rattled. Also, check your brakes before you, you know, brake." - Dr. Joshua Zahl, 09/29/2023.
\end{nquote}

\section{Sub-sequences}
Recall that sequences (of real numbers) are a function \(f:\N\to\R\); a more practical way to of defining the notation for sequences is \(x_n\) for \(x\in\N\). This is because we are effectively ``enumerating" the sequence, clarifying that it maps from the natural numbers. 
\begin{ndef}{: Sub-sequences}
	If \(x:\N\to\R\) is a sequence, then a \emph{\textbf{sub-sequence}} of the \(x\) is a sequence of the form \(x\circ g:\N\to\R\), where \(g:\N\to\N\) is \emph{strictly increasing}.
\end{ndef}
\begin{notation}
	This notation is clunky to work with, and the notation used in practice is \(x_{n_1},x_{n_2},x_{n_3},\dots,x_{n_k}\) such that \(n_1<n_2<n_3<\dots\)
\end{notation}

\clearpage

\section{Completeness (Dr. Zahl)}
\begin{nproperty}{ (a): Metric completeness}
	Every \emph{Cauchy sequence} of real numbers \emph{converges}.
\end{nproperty}
\begin{note}
	Once we define a metric it becomes more apparent why this is useful; currently, both convergent and Cauchy are the same, but when we depart from the real numbers this fact is less obvious and needs some more work.
\end{note}
\begin{nproperty}{ (b): Order completeness (Least upper bound property)}
	Consider \(\mc{S}\subseteq\R\) (\(\mc{S}\neq\emptyset\)); we define \(\mc{B}=\{b\in\R\st\text{for all}~s\in\mc{S},~s\leq b\}\). Either \(\mc{B}=\emptyset\) or \(\mc{B}=[\beta,\infty)\) for some \(\beta\in\R\).
\end{nproperty}

\textbf{Note}:

\begin{enumerate}[(a)]
	\item Every bounded sets have infinitely many upper bounds, however the least upper bound (supremum) tells us something about the structure of the set.
	
	\item The supremum is not necessarily in the set which it bounds.
	
	\item If we were to write the order completeness property for the rational numbers, this would not work. This is apparent if we define as \(\mc{S}\) for the rationals, we will see that we get a contradiction because there will always be a rational number smaller than \(\beta\) which will be an upper bound for \(\mc{S}\), causing the supremum to never exist.
\end{enumerate}

\begin{nproperty}{ (c): Monotone convergence property}
	If \((x_n)\) is \emph{monotone increasing}, either \(x_n\to\infty\), or \((x_n)\) converges.
\end{nproperty}
\begin{ntheorem}{}
	Properties (a), (b), (c) are equivalent
\end{ntheorem}
We have already shown that (a)\(\implies\)(b). We proceed by showing that (b)\(\implies\)(c):
\begin{proof}
	Let \((x_n)\) be a monotone increasing sequence. Let \(\mc{S}=\{x_n\st n\in\N\}\). If \(\mc{S}\) is \emph{not} bounded above, then \(x_n\to\infty\). Otherwise, the set of upper bounds \(\mc{B}\) is of the form \(\mc{B}=[\beta,\infty)\). Let us show that \(x_n\to\beta\).
	
	\medskip
	
	We know \(x_n\leq\beta\) for every \(n\). Let \(\eps>0\); there exists some \(x_N\in\mc{S}\) such that \(x_N>\beta-\eps\). Notice that this has to be true since if it wasn't, we would have \(x_n\leq \beta-\eps\), i.e., \(\beta-\eps\) is an upper bound for \(x_n\). However, since \(\beta\) is the supremum, and \(\beta-\eps<\beta\), this is clearly not true. Thus, for all \(n\geq N\), \(x_n\geq x_N>\beta-\eps\), i.e., \(|x_n-\beta|<\eps\).
\end{proof}
Now, we show that (c)\(\implies\)(a):
\begin{proof}
	The sketch for this proof is as follows:
	
	\medskip
	
	We begin by showing the following lemmas:
	\begin{nlemma}{-1}
		Every sequence of real numbers has a \emph{weakly} monotone sub-sequence.
	\end{nlemma}
	This is fairly intuitive; consider \(x_n=(-1)^n n\). This has uncountably many sub-sequences.
	\begin{nlemma}{-2}
		Every \emph{Cauchy sequence} is \emph{bounded}.
	\end{nlemma}
	The proof for this is basically pushing around definitions from before.
	\begin{nlemma}{-3}
		If \((x_n)\) is Cauchy and \((x_{n_k})\) is a sub-sequence and \(x_{n_k}\to\hat{x}\), then \(x_n\to\hat{x}\).
	\end{nlemma}
	To prove this we do require the fact that the sequence is Cauchy; every Cauchy sequence is bounded and from Lemma-1 we have a monotone increasing sub-sequence. We know this converges by the monotone convergence property, and thus, this forces the sequence itself to converge. So we combine all 3 lemmas to prove the claim.
\end{proof}