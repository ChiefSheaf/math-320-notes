\begin{ntheorem}{: The root test}
	Consider \(S=\sum_{n\in\N}a_n\), we let \(\alpha:=\displaystyle\limsup_{n\to\infty}|a_n|^{1/n}\)
	\begin{enumerate}[(a)]
		\item If \(\alpha<1\), then \(S\) converges absolutely.
		
		\item If \(\alpha>1\), then \(S\) diverges.
	\end{enumerate}
\end{ntheorem}
\begin{enumerate}[(a)]
	\item 
	\begin{proof}
		Given \(\alpha<1\), pick \(r\in(\alpha,1)\), for all \(n\leq N\) \(|a_n|^{1/n}<r\), so 
		\begin{equation*}
			\sum_{n=N}^{\infty}|a_n|<\sum_{n=N}^{\infty}r^n<1\quad(\text{geometric series}).
		\end{equation*}
	\end{proof}
	
	\item 
	\begin{proof}
		Pick \(R\in(1,\alpha)\implies|a_n|^{1/n}>R\) for infinitely many \(n\), \(|a_n|>R^n>1\), for all those \(n\), \(S\) diverges by crude test.
	\end{proof}
\end{enumerate}

\begin{ntheorem}{: Ratio test}
	Consider \(\displaystyle\sum_{n\in\N}a_n\) with \(a_n\neq 0\), for all \(n\)
	\begin{enumerate}[(a)]
		\item If \(\ol{\alpha}=\displaystyle\limsup_{n\to\infty}\displaystyle\frac{|a_{n+1}|}{|a_n|}<1\), then \(S\) converges absolutely.
		
		\item If \(\ul{\alpha}=\displaystyle\liminf_{n\to\infty}\displaystyle\frac{|a_{n+1}|}{|a_n|}>1\), then \(S\) diverges.
	\end{enumerate}
\end{ntheorem}
\begin{enumerate}[(a)]
	\item 
	\begin{proof}
		Choose \(r\in(\ol{\alpha},1)\). Since \(r<\ol{\alpha}\), there exists \(N\in\N\) such that for all \(n\geq N\), we have \(\displaystyle\frac{|a_{n+1}|}{|a_n|}<r\iff |a_{n+1}|<r|a_n|\). So, \(|a_{n+k}|<r|a_{n+k-1}|<\dots<r^k|a_n|\) for all \(k\in\N\). Thus,
		\begin{equation*}
			\sum_{k\in\N}|a_{N+k}|<|a_N|\sum_{k\in\N}r^k<+\infty;
		\end{equation*}
		this implies absolute convergence.
	\end{proof}
	
	\item 
	\begin{proof}
		The proof for this is left as an exercise.
	\end{proof}
\end{enumerate}

\subsection{Comparing these tests}
Given \(S=\displaystyle\sum_{n\in\N}a_n\), define \(\alpha:=\displaystyle\limsup_{n\to\infty}|a_n|^{1/n}\), \(\ol{\alpha}\) and \(\ul{\alpha}\) as above
\begin{enumerate}[(i)]
	\item \(\ol{\alpha}<1\implies \alpha<1\implies \displaystyle\sum_{n\in\N}|a_n|<+\infty\).
	
	\item \(\ul{\alpha}>1\implies \alpha>1\implies S\) diverges.
	
	\item If \(\alpha=1\implies (\ul{\alpha}\leq 1\leq\ol{\alpha})\), anything can happen.
\end{enumerate}
Let us prove the first implication of (i):
\begin{proof}
	When \(\ol{\alpha}=+\infty\), we are done.
	
	\medskip
	
	When \(\ol{\alpha}<+\infty\), it suffices to show that for all \(\eps>0\), \(\alpha\leq \ol{\alpha}+\eps\). Fix some \(\eps'>0\) and define \(\beta=\ol{\alpha}+\eps'\). Now, \(\beta>\ol{\alpha}\); there exists \(N\in\N\) such that \(n\geq N\), 
	\begin{equation*}
		\frac{|a_{n+1}|}{|a_n|}<\beta\implies |a_{n+1}|\leq\beta|a_n|.
	\end{equation*}
	For \(p\in\N\), we have
	\begin{equation*}
		|a_{N+p}|<\beta|a_{N+p-1}|<\dots<\beta^p|a_N|,
	\end{equation*}
	so for all \(m>N\in\N\), we have \(|a_m|^{1/m}<(\beta^{-N}|a_N|)^{1/m}(\beta^m)^{1/m}\). Therefore, we have 
	\begin{equation*}
		\limsup_{m\to\infty}|a_m|^{1/m}\leq\beta.
	\end{equation*}
\end{proof}
\begin{ntheorem}{: Cauchy condensation}
	If \(a_n\geq a_{n+1}\geq 1\), for all \(n\in\N\), the following are equivalent:
	\begin{enumerate}[(a)]
		\item \(S=\displaystyle\sum_{n=N}a_n<+\infty\).\\
		\item  \(T=\displaystyle\sum_{k\in\N}2^ka_{2^k}<+\infty\).
	\end{enumerate}
\end{ntheorem}
We start by showing that (b)\(\implies\)(a):
\begin{proof}
	Notice that 
	\begin{align*}
		a_1+a_2+\dots+a_n=&(a_1)+(a_2+a_3)+(a_4+a_5+a_6+a_7)+\dots\\
		\leq&a_1+2a_2+4a_4+\dots\\
		\leq& T<+\infty;
	\end{align*}
	this holds for all \(n\).
\end{proof}