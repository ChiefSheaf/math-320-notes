\begin{nquote}{}
	Some girl: ``Are you going to expect this on the exam?"
	
	\smallskip
	
	Dr. Loewen: ``Yes."
	
	\smallskip
	
	Girl: ``Oh no." - 22/09/2023
\end{nquote}

Recall that when we say that a limit \emph{diverges}, we are saying that for all \(\hat{x}\in\R\), there exists \(\eps>0\) and for all \(n\in\N\), there exists \(n>N\) such that \(|x_n-\hat{x}|\geq\eps\).
\begin{example}
	Show that \(x_n=(-1)^n\) diverges.
\end{example}
\begin{proof}
	To begin, we fix some \(\hat{x}\in\R\) and pick \(\eps=1\). Fix \(N\in\N\). Now, consider an even \(n_e\) and an odd \(n_o\) such that \(n_e>N,~n_o>N\). Thus, \(x_{n_e}=(-1)^{n_e}=1\) and \(x_{n_o}=(-1)^{n_o}=-1\). Thus, 
	\begin{align*}
		2=|x_{n_e}-x_{n_o}|\leq&|(x_{n_e}-\hat{x})+(\hat{x}-x_{n_o})|\\
		\leq&|x_{n_e}-\hat{x}|+|x_{n_o}-\hat{x}|.
	\end{align*}
	One of the terms on the RHS is \(\geq 1\). One of \(n=n_e\) or \(n=n_o\) completes the proof.
\end{proof}

\clearpage

\section{The Squeeze Theorem}
Let \((a_n),~(x_n),~(b_n)\) be real-valued sequences, and \(L\in\R\). Assume
\begin{enumerate}[(a)]
	\item \(a_n\to L\) as \(n\to\infty\).
	
	\item \(b_n\to L\) as \(n\to\infty\).
	
	\item \(a_n\leq x_n\leq b_n\) for all \(n>N\).
\end{enumerate}
Then, \(x_n\to L\) as \(n\to\infty\).
\begin{proof}
	Given \(\eps>0\), use (a) to get \(N_a\in\N\) such that \(|a_n-L|<\eps\) for all \(n>N_a\). This implies \(a_n>L-\eps\). Use (b) to get \(N_b\in\N\) such that \(|b_n-L|<\eps\) for all \(n>N_b\). This implies \(b_b<L+\eps\). Use (c) to get \(N_c\in\N\) such that \(a_n\leq x_n\leq b_n\) for all \(n>N_c\). Now, if \(N=\operatorname{max}\{N_a,N_b,N_c\}\), every \(n>N\) does 3 things: \(L-\eps< a_n\leq x_n\leq b_n<L+\eps\), i.e., \(|x_n-L|<\eps\).
\end{proof}