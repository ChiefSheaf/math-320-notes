\(\mc{G}\st \mc{X}\rightrightarrows \mc{Y}\) is built as a subset \(\mc{G}\) of \(\mc{X}\times \mc{Y}\). Denote 
\begin{equation*}
	\mc{G}(x)=\{y\in \mc{Y}\st (x,y)\in \mc{G}\}.
\end{equation*}
We go on to invent a \(\mc{G}^{-1}=\{(y,x)\st (x,y)\in \mc{G}\}\subseteq \mc{Y}\times \mc{X}\) gives a related \(\mc{G}^{-1}\st \mc{Y}\rightrightarrows \mc{X}\) such that 
\begin{ndef}{: Pre-image}
	\begin{align*}
		\mc{G}^{-1}(y)=&\{x\st (y,x)\in \mc{G}^{-1}\}\\
		=&\{x\st (y,x)\in \mc{G}\}\\
		=&\{x\st y\in \mc{G}\}.
	\end{align*}
	This is called the \emph{\textbf{pre-image of y}}.
\end{ndef}
Now, define \(\operatorname{dom}(\mc{G})=\{x\in \mc{X}\st \mc{G}(x)\neq \phi\}\). Extend this notation to inputs that are \emph{sets}:
\medskip
if \(\mc{A}\subseteq \mc{X}\)
\begin{equation*}
	\mc{G}(\mc{A})=\bigcup_{a\in \mc{A}}\mc{G}(a),
\end{equation*} 
if \(\mc{B}\subseteq \mc{Y}\),
\begin{align*}
	\mc{G}^{-1}(\mc{B})=&\bigcup_{b\in \mc{B}}\mc{G}^{-1}(b)\\
			 =&\{x\in \mc{X}\st \mc{G}(x)\cap \mc{B}\neq \emptyset\}.
\end{align*}

\medskip

A set-valued map \(\mc{G}\st \mc{X}\rightrightarrows \mc{Y}\) is a \emph{mapping} or \emph{function} when \(\mc{G}(x)\) is a \emph{singleton} (one-point set) for each \(x\in \mc{X}\) (this entails \(\operatorname{dom}(\mc{G})=\mc{X}\).) 

\begin{notation}
	Indicate  this by writing \(\mc{G}\st \mc{X}\to \mc{Y}\), and simplifying \(\mc{G}(x)=\{y\}\) down to \(\mc{G}(x)=y\).
\end{notation}

\begin{note}
	A set \(\mc{G}\) might give a function, but \(\mc{G}^{-1}\) might not.
\end{note}

\begin{example}
	Consider the equation \(y=x^2\). Here, \(\mc{X}=\R\), \(\mc{Y}=\R\), and \(\mc{X}\times \mc{Y}=\R^2\). The equation defines a set \(\mc{G}=\{(x,y)|~y=x^2\}\) in the plane that is the graph of the function \(\mc{G}:\R\to \R\) defined by \(\mc{G}(t)=t^2\), \(t\in \R\).
	\medskip
	Note that 
	\begin{align*}
		\mc{G}^{-1}(4)=&\{x\st \mc{G}(x)=4\}\\
		=&\{-2,+2\}.
	\end{align*}
	Similarly,
	\begin{align*}
		\mc{G}^{-1}(0)=&\{x\st \mc{G}(x)=0\}\\
		=&\{-0,0\}=0.
	\end{align*}
	Also, \(\operatorname{dom}(\mc{G}^{-1})=\{y\st y\geq 0\}=[0,\infty)\).
\end{example}
Thus, we go on to define a function:
\begin{ndef}{: Function}
	Let \(f\st \mc{X}\to \mc{Y}\),
	\begin{enumerate}
		\item This \(f\) is one-to-one (injective), i.e., different inputs give different outputs, i.e.,
		\begin{itemize}
			\item for all \(x_1,~x_2\in \mc{X}\), \(x_1\neq x_2\) if \(f(x_1)\neq f(x_2)\).
			
			\item We can also state the contrapositive: for all \(x_1,~x_2\in \mc{X}\), \(f(x_1)=f(x_2)\implies x_1=x_2\).
		\end{itemize}
		
		\item This \(f\) is \emph{onto} (surjective) when \(f(\mc{X})=\mc{Y}\), i.e., 
		\begin{itemize}
			\item for all \(y\in \mc{Y}\), there exists \(x\in \mc{X}\) such that \(f(x)=y\).
		\end{itemize}
		
		\item A function that is \emph{both} one-to-one and onto is called \emph{bijective}.
	\end{enumerate}
\end{ndef}

\clearpage

\section{Countability}
Recall that \(\N=\{0,1,2,3,\dots\}\), are defined to be the natural numbers. 
\begin{ndef}{: Finite sets}
	A set \(\mc{A}\) is \emph{\textbf{finite}} if either \(\mc{A}=\phi\) or there exists \(n\) and a bijection \(\varphi\st \{1,2,\dots, n\}\to \mc{A}\).
\end{ndef}

\begin{notation}
	\(|\emptyset|=0\), \(|\mc{A}|=n\).
\end{notation}

This is more or less as expected, but \emph{countable} sets are more interesting.
\begin{ndef}{: Countable}
	A set \(\mc{S}\) is \emph{\textbf{countable}} if there is a bijection \(\varphi:~\N\to \mc{S}\). Here, \(|\mc{S}|=\aleph_0\).
	\begin{example}
		Consider
		\begin{enumerate}
			\item \(\N\) itself; \((\varphi(x)=x)\).
			
			\item Hilbert's hotel: \(S=\N\cup\{0\}\); use \(\varphi(n)=n-1\).
			
			\item \(\mathbb{Z}=\{\dots,-2,-1,0,1,2,\dots\}\). \newline
			Use 
			\begin{equation*}
				\phi(n)=\begin{cases}
					\displaystyle-\frac{n}{2}&\text{if \(n\) is even}.\\\\
					\displaystyle\frac{n-1}{2}&\text{if \(n\) is odd}.
				\end{cases}
			\end{equation*} 
		\end{enumerate}
	\end{example}
\end{ndef}

