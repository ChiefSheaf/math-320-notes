\begin{nquote}{}
	"That's a little informal; I got tired of writing."- Philip Loewen, 09/13/2023
	
	\medskip
	
	"Do they still teach LISP?" - Philip Loewen, 09/13/2023
\end{nquote}

\begin{claim}
	The union of countably many sets, each one countable is also countable.
\end{claim}
\begin{proof}
	Let \(\mc{A}^{(1)},~\mc{A}^{(2)},~\mc{A}^{(3)},\dots\) be a (countable) family of sets, each countable. This means each \(k\) comes with some \(\varphi_k\st\N\to \mc{A}^{(k)}\), a bijection. So, define,
	\begin{equation*}
		f\st\N\times\N \to \bigcup_{k=1}^{\infty}\mc{A}^{(k)},
	\end{equation*}
	by \(f(m,n)=\varphi_m(n)\). This is \emph{surjective}, and input set \(\N\times\N\) is countable; we are done.
\end{proof}

\begin{claim}
	The set \(\N\times\N\times\N\) is countable.
\end{claim}
\begin{proof}
	By definition,
	\begin{equation*}
		\N\times\N\times\N=\{(x_1,~x_2,~x_3)\st x_k\in\N\}
	\end{equation*}
	is the image of 
	\begin{equation*}
		(\N\times\N)\times\N=\{((x_1,~x_2),~x_3)\st x_1,~x_2,~x_3\in\N\}
	\end{equation*}
	under \(\varphi((x_1,~x_2),~x_3)=(x_1,~x_2,~x_3)\). 
	Clearly, \(\varphi\) is bijective and \((\N\times\N)\times\N\) is the cartesian product of two countable sets.
	
	\medskip
	
	Extend by induction:
	
	\smallskip
	
	For any \(n\in\N\), the n-fold product
	\begin{equation*}
		\prod_{k=1}^{n}\N=\underbrace{\N\times\N\times\N\times\dots\times\N}_{n-\text{copies}}
	\end{equation*}
	is countable. Thus,
	\begin{equation*}
		\bigcup_{n\in\N}\left(\prod_{k=1}^{n}\N\right)
	\end{equation*}
	is countable. These are all the finite-length tuples with entries from \(\N\).
\end{proof}
\begin{ndef}{: Uncountable}
	A set \(\mc{Y}\) is \emph{\textbf{uncountable}} iff \(\mc{Y}\) is infinite and not countable, i.e., \(\mc{Y}\) is infinite and every function \(\varphi:\N\to \mc{Y}\) fails to be bijective.
\end{ndef}
\begin{note}
	We have to be careful with this definition. This is how Rudin defines uncountability,where exclusively infinite sets to be countable. For intuition, uncountable sets are all sets that are not countable.
\end{note}
\begin{note}
	To prove that a set is \(\mc{Y}\) is uncountable, we show that every \(\varphi:\N\to \mc{Y}\) that is injective cannot be surjective.
\end{note}
\begin{example}
	Let \(\mc{I}\) be the set of sequences having the form
	\begin{equation*}
		x=0.d_1d_2d_3\dots,
	\end{equation*}
	where each \(d_k\in\{0,~1,~2,\dots,~9\}\) and for all \(n\in\N\) there exists \(m\geq n:~d_m\neq 9\) (digit strings that do not ``end" with an infinite list of 9's.) This \(\mc{I}\) is uncountable. To prove this, pick some \(\varphi:\N\to\mc{I}\) that is injective. We shall show that \(\varphi\) is not surjective (this was first done by Cantor.) We note the list of \(\pfi\) values:
	\begin{align*}
		\varphi(1)=&0.\boxed{d_1^{(1)}}d_2^{(1)}d_3^{(1)}d_4^{(1)}\dots\\
		\varphi(2)=&0.d_1^{(2)}\boxed{d_2^{(2)}}d_3^{(2)}d_4^{(2)}\dots\\
		\varphi(3)=&0.d_1^{(3)}d_2^{(3)}\boxed{d_3^{(3)}}d_4^{(3)}\dots\\
		\vdots&~~~~~~~~~~~~~~~~~~~~~~~~~~~~~\ddots\\
	\end{align*}
	Now, for each \(k\in\N\), invent 
	\begin{equation*}
		x_k=\begin{cases}
			5&\text{if}~d_k^{(k)}=7\\
			7&\text{else}.
		\end{cases}
	\end{equation*}
	(Keep \(x_i=d_i^{(k)}\) where \(i\neq k\).)
	
	\medskip
	
	Consider \(x:=0.x_1x_2x_3\dots\). Now, \(x\notin\varphi(\N)\) because \(x\) differs in position \(k\) from the string \(\varphi(k)\) so \(\varphi\) is not surjective.\qed
	\begin{note}
		Technically, we do not need to require that \(\varphi\) is injective.
	\end{note}
	\begin{note}
		Use \(\R\) as known for now; we will build axiomatic connections later.
	\end{note}
	In this case, each \(x=0.x_1x_2x_3\dots\) in \(\mc{I}\) defines the real number 
	\begin{equation*}
		\sum_{k=1}^{\infty}\frac{x_k}{10^k},
	\end{equation*}
	and the proof above shows \(\mc{I}=[0,1)\) is uncountable.
\end{example}