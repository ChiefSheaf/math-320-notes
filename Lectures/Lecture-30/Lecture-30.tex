Now \(\hat{\mc{X}}=\{P[x]\st x\in\operatorname{CS}(\mc{X})\}\) with metric \(D\) is a \emph{complete} space implies that every Cauchy sequence converges. To show this, we start with a Cauchy sequence in \((\hat{\mc{X}},D)\), say \(A_1,A_2,\dots\), i.e., \(A_p=P[x^{(p)}]\) for some \(x^p\in\operatorname{CS}(\mc{X})\), i.e., \(x^{(p)}=(x_1^{(p)},x_2^{(p)},x_3^{(p)},\dots)\). To show that \((A_p)\) converges in \((\hat{\mc{X}},D)\), we first identify a candidate \(\hat{A}\in\hat{X}\) for the limit. We need \(\hat{A}\in P[\hat{x}]\) for some \(\hat{x}\in\operatorname{CS}(\mc{X})\), so we build that. We use the \(\operatorname{CS}(\mc{X})\) property for each sequence \(x^{(1)},x^{(2)},x^{(3)},\dots\).
\begin{align*}
	\text{Pick a sufficiently large}~&n_1~\text{such that}~d(x_j^{(1)},x_k^{(1)})<1,~ \text{for all}~j,k>n_1\\
	\text{Pick a sufficiently large}~&n_2>n_1~\text{such that}~d(x_j^{(2)},x_k^{(2)})<\frac{1}{2},~ \text{for all}~j,k>n_2\\
	\text{Pick a sufficiently large}~&n_3>n_2~\text{such that}~d(x_j^{(3)},x_k^{(3)})<\frac{1}{3},~ \text{for all}~j,k>n_3\\
	&\vdots
\end{align*}
Each stage gives an element of \(\hat{x}\):
\begin{align*}
	\hat{x}_1&=x_{n_1}^{(1)}\\
	\hat{x}_2&=x_{n_2}^{(2)}\\
	&\vdots\\
	\hat{x}_p&=x_{n_p}^{(p)}\\
	&\vdots
\end{align*}
\begin{claim}
	Sequence \(\hat{x}_p\) is Cauchy.
\end{claim}
\begin{proof}[Proof sketch]
	The bulk of the proof is left as an exercise. Start with 
	\begin{align*}
		d(\hat{x}_p,\hat{x}_q)=d(x_{n_p}^{(p)},x_{n_q}^{(q)})&\leq d(x_{n_p}^{(p)},x_j^{(p)})+d(x_j^{(p)},x_j^{(q)})+d(x_j^{(q)},\hat{x}_{n_q}^{(q)})\\
		&\leq \frac{1}{p}+d(x_j^{(p)},x_j^{(q)})+\frac{1}{q}
	\end{align*}
	provided \(j\geq\operatorname{max}\{n_p,n_q\}\). Consider limit as \(j\to\infty\), which tells us 
	\begin{equation*}
		d(\hat{x}_p,\hat{x}_q)\leq \frac{1}{p}+D(P[x^{(p)}],P[x^{(q)}])+\frac{1}{q}\dots
	\end{equation*}
\end{proof}
\begin{claim}
	\(\hat{A}=P[\hat{x}]\) obeys \(\displaystyle\lim_{p\to\infty}A_p=\hat{A}\), i.e., \(D(A_p,\hat{A})\to 0\) as \(p\to\infty\).
\end{claim}
\begin{proof}
	Left as an exercise.
\end{proof}
\begin{note}
	After the construction succeeds, think of an original \(\mc{X}\) as a subset of \(\hat{\mc{X}}\) (true embedding is \(\{\Phi[x]\st x\in\mc{X}\}\), but they are functionally indistinguishable). Then \(\ol{\mc{X}}=\hat{X}\). However, before \(\hat{\mc{X}}\) is built, original \(\mc{X}\) is closed as a subset of \((\mc{X},d)\), so in that \emph{original} setup, \(\ol{\mc{X}}=\mc{X}\).
\end{note}

\subsection{Cantor set}
Consider the subset of \(C_0=[0,1]\) that we obtain by throwing away the middle third \(\Omega_0=\left(\displaystyle\frac{1}{3},\frac{2}{3}\right)\), i.e., \(C_1=C_0\backslash\Omega_0\). This \(C_1\) has \(2\) closed intervals: let \(\Omega_1\) be the \(2\) open middle third pieces: \(\Omega_1=\left(\displaystyle\frac{1}{9},\frac{2}{9}\right)\cup\left(\displaystyle\frac{7}{9},\frac{8}{9}\right)\). We have \(C_2=C_1\backslash\Omega_1\), and we keep doing this.
\begin{figure}[H]
	\centering
	\scalebox{1.4}{%
		\begin{tikzpicture}[decoration=Cantor set,very thick]
			\draw[-, very thick] (0,0.5) -- (7,0.5);
			\node[left] at (-0.75,0.5) {\(C_0:\)}; 
			\draw decorate{ (0,0) -- (7,0) };
			\node[left] at (-0.75,0) {\(C_1:\)};
			\draw decorate{ decorate{ (0,-.5) -- (7,-.5) }};
			\node[left] at (-0.75,-0.5) {\(C_2:\)};
			\draw decorate{ decorate{ decorate{ (0,-1) -- (7,-1) }}};
			\node[left] at (-0.75,-1) {\(C_3:\)};
			\node[left] at (-1.1,-1.5) {\(\vdots\)};
			\node[left] at (0.58,-1.5) {\(\vdots\)};
			\node[left] at (2.14,-1.5) {\(\vdots\)};
			\node[left] at (5.26,-1.5) {\(\vdots\)};
			\node[left] at (6.82,-1.5) {\(\vdots\)};
			\draw[-{Stealth[scale=0.8]}, thick, color=blue] (3.5,0.5) -- (1.17,0);
			\draw[-{Stealth[scale=0.8]}, thick, color=blue] (1.17,0) -- (1.94,-0.5);
			\draw[-{Stealth[scale=0.8]}, thick, color=blue] (1.94,-0.5) -- (1.68,-1);
			\draw[-{Stealth[scale=0.8]}, thick, color=blue] (1.68,-1) -- (1.77,-1.5);
			\node[above] at (0.05,0.5) {\(0\)};
			\node[above] at (6.95,0.5) {\(1\)};
		\end{tikzpicture}
	}
	\caption{Visualization of the middle-thirds Cantor set; the blue path is explained in part (ii) of the notable properties of the cantor set.}
\end{figure}
The cantor set is defined as 
\begin{equation*}
	\mc{C}:=\bigcap_{k=0}^{\infty}C_k,
\end{equation*}
which is considered a very rich example in analysis.

\smallskip

Some notable properties of the cantor set:
\begin{enumerate}[(i)]
	\item \(C\neq\emptyset\) since clearly \(0,1\in\mc{C}\). By self-similarity, endpoints of closed intervals in set \(C_k\) \emph{all} lie in \(\mc{C}\).
	
	\item  \(|\mc{C}|=|\R|\implies\mc{C}\) is uncountable. This is because any \(0-1\) sequence defines a left-right path down the tree, as shown in the diagram above, that selects a nested sequence of closed intervals with a \(1-\)point intersection. Different sequences select different points of \(\mc{C}\) (number of \(0-1\) sequences equals \(|\R|\)).
	
	\item \(\mc{C}'=\mc{C}\).
	
	\item \(\mc{C}^{\circ}=\emptyset\).
	
	\item The total length of open sets removed from \([0,1]\) to define \(\mc{C}\) equal \(1\). If length has any meaning for set \(\mc{C}\), the only possible value is \(0\).
\end{enumerate}