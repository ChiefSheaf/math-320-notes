\begin{nquote}{}
	``Let's back up and see how nice it would be to be so smart that you're right all the time." - Dr. Philip Loewen, 12/06/2023
	
	\medskip
	
	``A tragedy, the butterfly! We killed the butterfly! Ah, it'll come back in the spring." - Dr. Philip Loewen, 12/06/2023
	
	\medskip
	
	``Job done." - Dr. Philip Loewen, 12/06/2023
	
	\medskip
	
	``The final exam, it's  worth a lot right? It's worth 50\%, so you probably want to study." - Dr. Philip Loewen, 12/06/2023
\end{nquote}

Some common misconceptions:

\begin{enumerate}
	\item The tangent line touches the graph at one point only.
	
	\begin{figure}[H]
		\centering
		\begin{tikzpicture}
			\draw[Latex-Latex, thick] (-3,0)--(3,0);
			
			\node[below] at (3,0) {\(x\)};
			
			\draw[Latex-Latex, thick] (0,-3)--(0,3);
			
			\node[left] at (0,3) {\(y\)};
			
			\draw[-, thick, red] (-0.9, -2.5) to [curve through = {(-0.5,0)..(0,1.5)..(0.5,2)..(1.3,0.8)..(1.7,1.5)}] (2,2);
			
			\draw[-, thick, samples = 100, domain = -2:3, variable = \x] plot ({\x},{\x - 0.64});
			
			\draw[fill] (1.45,0.85) circle (1pt);
			
			\draw[-, thick, dashed] (1.45,0.85)--(1.45,0);
			
			\node[below] at (1.45,0) {\(c\)};
		\end{tikzpicture}
		\caption{Example where the tangent line at a point passes through the curve at another point.}
	\end{figure}
	
	\clearpage
	
	\item The tangent line touches the curve but doesn't cross it.
	\begin{figure}[H]
		\centering
		\begin{tikzpicture}
			\draw[Latex-Latex, thick] (-4,0)--(4,0);
			
			\node[below] at (4,0) {\(x\)};
			
			\draw[Latex-Latex, thick] (0,-2)--(0,4);
			
			\node[left] at (0,4) {\(y\)};
			
			\draw[-, thick, samples = 100, domain = -1.1:1.2, variable = \x, red] plot ({\x},{tan((\x + 3) r) + 2});
			
			\draw[-, thick, samples = 100, domain = -3:3, variable = \x] plot ({\x},{0.7*\x + 1.875});
		\end{tikzpicture}
		\caption{Example where the tangent line touches the curve and crosses it.}
	\end{figure}
	
	\item If \(f'(c)>0\), then \(f\) is increasing on some open interval with midpoint \(c\). Consider the example
	\begin{equation*}
		f(x)=\begin{cases}
				\frac{1}{2}x+x^2\sin{\left(\frac{1}{x}\right)}&\text{if}~x\neq 0\\
				0&\text{at}~x=0.			
			 \end{cases}
	\end{equation*}
	Note that 
	\begin{equation*}
		f'(x)=\frac{1}{2}-\cos{\left(\frac{1}{x}\right)}+2x\sin{\left(\frac{1}{x}\right)},
	\end{equation*}
	has positive values in the interval \((-\delta,\delta)\).
\end{enumerate}

\subsection{Optimization}

\begin{ntheorem}{}
	Suppose \(f:(a,b)\to\R\) has a minimum at some point \(c\in(a,b)\), i.e., \(f(c)\leq f(x)\) for all \(x\in(a,b)\), then \(f'(c)=0\).
\end{ntheorem}

\begin{proof}
	We will show this proof by contrapositive, and use butterfly lemma.
	
	\medskip
	
	Suppose \(f'(c)\neq 0\). Without loss of generality, assume \(f'(c)>0\). Pick \(m=0\), and use butterfly lemma to get \(\delta>0\) such that 
	\begin{equation*}
		f(x)<f(c),~\text{for all}~x\in (c-\delta,c)
	\end{equation*} 
	Thus, \(f(c)\) is \emph{not} a minimizer on \((a,b)\).
\end{proof}

\begin{ntheorem}{: (Darboux)}
	Derivatives have the intermediate value property:
	
	\medskip
	
	If \(f\) is differentiable at all points of the closed interval \([a,b]\) and \(\mu\) lies between \(f'(a)\) and \(f'(b)\), then there exists \(c\in (a,b)\) where \(f'(c)=\mu\).
\end{ntheorem}

\begin{proof}
	The result is obvious if \(f'(a)=\mu\) or \(f'(b)=\mu\).
	
	\medskip
	
	Suppose without loss of generality, \(f'(a)<\mu<f'(b)\). Consider the new function
	\begin{equation*}
		g(x)=f(x)-\mu x:
	\end{equation*}
	this is a continuous function on \([a,b]\) with
	\begin{align*}
		&g'(a)=f'(a)-\mu\\
		&g'(b)=f'(b)-\mu.
	\end{align*}
	Hence, by butterfly lemma, that the absolute minimum of function \(g\) on the set \([a,b]\) cannot occur at either end. So, it must be at some point \(c\in (a,b)\), and thus, \(0=g'(c)=f'(c)-\mu\).
\end{proof}

\begin{ntheorem}{: Mean Value Theorem}
	Given \(f:[a,b]\to\R\) is a continuous function, suppose \(f'(x)\) exists for all \(x\in (a,b)\). Then, there exists \(c\in(a,b)\) such that \(\displaystyle\frac{f(b)-f(a)}{b-a}=f'(c)\).
\end{ntheorem}
\begin{figure}[H]
	\centering
	\begin{tikzpicture}[scale=1.5]
		\draw[Latex-Latex, thick] (-1,0)--(4,0);
		
		\node[below] at (4,0) {\(x\)};
		
		\draw[Latex-Latex, thick] (0,-1)--(0,4);
		
		\node[left] at (0,4) {\(y\)};
		
		\draw[-, thick, red] (1, 1) to [curve through = {(2.3,2.7)}] (3,3);
		
		\draw[fill] (1,1) circle (1pt);
		
		\node[] at (1,0) {\(\mid\)};
		
		\node[below] at (1,-0.1) {\(a\)};
		
		\node[] at (3,0) {\(\mid\)};
		
		\node[below] at (3,-0.1) {\(b\)};
		
		\draw[fill] (3,3) circle (1pt);
		
		\draw[-, thick, dashed] (1,1)--(3,3);
		
		\draw[-, thick, samples = 100, domain = -0.5:4, variable = \x] plot ({\x},{\x + 0.52});
		
		\draw[fill] (1.75,2.27) circle (1pt);
		
		\draw[-, dashed] (1.75,2.27)--(1.75,0);
		
		\node[below] at (1.75,0) {\(c\)};
	\end{tikzpicture}
	\caption{Visualization of the mean value theorem.}
\end{figure}

\begin{proof}
	Define \(m:=\displaystyle\frac{f(b)-f(a)}{b-a}\), and let \(g(x)=f(x)-m x\). Evaluate
	\begin{align*}
		&g(a)=f(a)-m a\\
		&g(b)=f(b)-m b.
	\end{align*}
	Note,
	\begin{align*}
		g(b)-g(a)=&[f(b)-f(a)]-m(b-a)\\
		 		 =&0.
	\end{align*}
	Since \(g\) is continuous with a compact domain, it has absolute maximum and absolute minimum \([a,b]\).
	
	\smallskip
	
	If \(g\) is constant, those are equal, but
	\begin{equation*}
		0=g'(c)=f'(c)-m,~\text{for all}~c\in (a,b).
	\end{equation*}
	
	\smallskip
	
	If \(g\) is not constant, then some \(c\in (a,b)\) will provide a local extremum and give 
	\begin{equation*}
		0=g'(c)=f'(c)-m.
	\end{equation*}
\end{proof}

\begin{corollary}
	If \(f'(x)>0\) for all \(x\in (a,b)\), then \(f\) is increasing on \((a,b)\).
\end{corollary}
\begin{proof}
	If \(\tilde{a}<\tilde{b}\) lie in \((a,b)\); MVT says 
	\begin{equation*}
		\frac{f(\tilde{b})-f(\tilde{a})}{\tilde{b}-\tilde{a}}=f'(c)>0,~\text{for some}~c\in(\tilde{a},\tilde{b}).
	\end{equation*}
	Hence, \(f(\tilde{b})>f(\tilde{a})\).
\end{proof}

\begin{center}
	\textbf{End of MATH 320 :-)}
\end{center}